\documentclass[10pt,twocolumn,letterpaper]{article}

\usepackage{cvpr}
\usepackage{times}
\usepackage{epsfig}
\usepackage{graphicx}
\usepackage{amsmath}
\usepackage{amssymb}
\usepackage{float}

% Include other packages here, before hyperref.

% If you comment hyperref and then uncomment it, you should delete
% egpaper.aux before re-running latex.  (Or just hit 'q' on the first latex
% run, let it finish, and you should be clear).
\usepackage[pagebackref=true,breaklinks=true,letterpaper=true,colorlinks,bookmarks=false]{hyperref}

\cvprfinalcopy % *** Uncomment this line for the final submission

\def\cvprPaperID{****} % *** Enter the CVPR Paper ID here
\def\httilde{\mbox{\tt\raisebox{-.5ex}{\symbol{126}}}}

% Pages are numbered in submission mode, and unnumbered in camera-ready
\ifcvprfinal\pagestyle{empty}\fi
\begin{document}

%%%%%%%%% TITLE
\title{Face detection in a video sequence - a temporal approach}

\author{K. Mikolajczyk\\
INRIA\\
Rh\^{o}ne-Alpes GRAVIR-CNRS, 655 av. de l’Europe, 38330 Montbonnot, France\\
{\tt\small Krystian.Mikolajczyk@inrialpes.fr}
% For a paper whose authors are all at the same institution,
% omit the following lines up until the closing ``}''.
% Additional authors and addresses can be added with ``\and'',
% just like the second author.
% To save space, use either the email address or home page, not both
\and
R. Choudhury\\
{\tt\small Ragini.Choudhury@inrialpes.fr}
\and
C. Schmid\\
{\tt\small Cordelia.Schmid@inrialpes.fr}
}

\maketitle
%\thispagestyle{empty}

%%%%%%%%% ABSTRACT
\begin{abstract}
This paper presents a new method for detecting faces in
a video sequence where detection is not limited to frontal
views. The three novel contributions of the paper are :
1) Accumulation of probabilities of detection over a sequence. This allows to obtain a coherent detection over
time as well as independence from thresholds. 2) Prediction of the detection parameters which are position, scale
and pose. This guarantees the accuracy of accumulation as
well as a continuous detection. 3) The way pose is represented. The representation is based on the combination of
two detectors, one for frontal views and one for profiles.
Face detection is fully automatic and is based on the
method developed by Schneiderman. It uses local histograms of wavelet coefficients represented with respect to
a coordinate frame fixed to the object.

......
\end{abstract}

%%%%%%%%% BODY TEXT
\section{Introduction}

In the context of video structuring, indexing and visual
surveillance, faces are the most important “basic units”. If
the application is, for example, to identify an actor in a
video clip, face detection is required as the first step.

Existing approaches either detect faces for every frame
without using the temporal information or they detect a face
in the first frame and then track the face through the sequence with a separate algorithm. This paper presents a
novel approach which integrates detection and tracking in
a unified probabilistic framework. It uses the temporal relationships between the frames to detect human faces in a
video sequence, instead of detecting them in each frame independently.

......

{\bf The rest of the paper is skipped here, and the text sections below are the assignments for the Week 2.}
%-------------------------------------------------------------------------

%-------------------------------------------------------------------------
\section{A exemple of citation}

Here is a exemple of citation\cite{Mikolajczyk2001Face}.


%------------------------------------------------------------------------
\section{One math formula}

$Var(S) = \sum_{i=1}^S w_i{S}^2-{Mean(S)}^2$

%-------------------------------------------------------------------------
\section{One table}

\begin{table}[H]
\begin{center}
\begin{tabular}{|l|c|}
\hline
Method & XXXX \\
\hline
Theirs & XXXX \\
Ours & XXXX\\
\hline
\end{tabular}
\end{center}
\caption{Example of table.}
\end{table}



%-------------------------------------------------------------------------
\section{One image}

\begin{figure}[H]
\begin{center}
\includegraphics[scale=0.5]{02.png}
\end{center}
   \caption{Example of image.}
\end{figure}

%-------------------------------------------------------------------------


%------------------------------------------------------------------------

{\small
\bibliographystyle{ieee_fullname}
\bibliography{egbib}
}

\end{document}
