\documentclass[10pt,twocolumn,letterpaper]{article}

\usepackage{cvpr}
\usepackage{times}
\usepackage{epsfig}
\usepackage{graphicx}
\usepackage{amsmath}
\usepackage{amssymb}
\usepackage{multirow}

% Include other packages here, before hyperref.

% If you comment hyperref and then uncomment it, you should delete
% egpaper.aux before re-running latex.  (Or just hit 'q' on the first latex
% run, let it finish, and you should be clear).
\usepackage[breaklinks=true,bookmarks=false]{hyperref}

\cvprfinalcopy % *** Uncomment this line for the final submission

\def\cvprPaperID{244} % *** Enter the CVPR Paper ID here
\def\httilde{\mbox{\tt\raisebox{-.5ex}{\symbol{126}}}}

% Pages are numbered in submission mode, and unnumbered in camera-ready
%\ifcvprfinal\pagestyle{empty}\fi
\setcounter{page}{1}
\begin{document}

%%%%%%%%% TITLE
\title{Learning Robust Visual-Semantic Embeddings\\}

\author{Xuepeng Shi\\
{\tt\small xuepeng.shi@vipl.ict.ac.cn}
% For a paper whose authors are all at the same institution,
% omit the following lines up until the closing ``}''.
% Additional authors and addresses can be added with ``\and'',
% just like the second author.
% To save space, use either the email address or home page, not both
\and
Shiguang Shan\\
{\tt\small shiguang.shan@vipl.ict.ac.cn}
}

\maketitle
%\thispagestyle{empty}

%%%%%%%%% ABSTRACT
\begin{abstract}
   Rotation-invariant face detection, i.e. detecting faces
   with arbitrary rotation-in-plane (RIP) angles, is widely required
   in unconstrained applications but still remains as a
   challenging task, due to the large variations of face appearances.
   Most existing methods compromise with speed or
   accuracy to handle the large RIP variations.To address
   this problem more efficiently, we propose Progressive Calibration
   Networks (PCN) to perform rotation-invariant face
   detection in a coarse-to-fine manner. PCN consists of three
   stages, each of which not only distinguishes the faces from
   non-faces, but also calibrates the RIP orientation of each
   face candidate to upright progressively. By dividing the
   calibration process into several progressive steps and only
   predicting coarse orientations in early stages, PCN can
   achieve precise and fast calibration. By performing binary
   classification of face vs. non-face with gradually decreasing
   RIP ranges, PCN can accurately detect faces with full
   360$^\circ$ RIP angles. Such designs lead to a real-time rotationinvariant
   face detector. The experiments on multi-oriented
   FDDB and a challenging subset of WIDER FACE containing
   rotated faces in the wild show that our PCN achieves
   quite promising performance. A demo of PCN can be available
   at \url{https://github.com/Jack-CV/PCN}.
\end{abstract}

%%%%%%%%% BODY TEXT
\section{Introduction}

Face detection serves as an important component in computer
vision systems which aim to extract information from
face images. Practical applications, such as face recognition
and face animation, all need to quickly and accurately detect
faces on input images in advance. Same as many other
vision tasks, the performance of face detection has been
substantially improved by Convolutional Neural Network(CNN)
[\cite{Alpher02},\cite{Alpher03}].The CNN-based
detectors enjoy the natural advantage of strong capability
in non-linear feature learning. However, most works focus
on designing an effective detector for generic faces without
considerations for specific scenarios, such as detecting faces
with full rotation-in-plane (RIP) angles as shown in Figure
1. They become less satisfactory in such complex applications.
Face detection in full RIP, i.e. rotation-invariant
face detection, is quite challenging, because faces can be
captured almost from any RIP angle, leading to significant
divergence in face appearances. An accurate rotationinvariant
face detector can greatly boost the performance of
subsequent process, e.g. face alignment and face recognition.

\begin{figure}[t]
\begin{center}
%\fbox{\rule{0pt}{2in} \rule{0.9\linewidth}{0pt}}
   \includegraphics[width=0.8\linewidth]{figure1.jpg}
\end{center}
   \caption{Many complex situations need rotation-invariant face detectors.
The face boxes are the outputs of our detector, and the blue
line indicates the orientation of faces.}
\label{fig:long}
\label{fig:onecol}
\end{figure}

%-------------------------------------------------------------------------
\subsection{PCN-1 in 1st stage}
For each input window $\mathit{x}$, PCN-1 has three objectives:
face or non-face classification, bounding box regression,
and calibration, formulated as follows:
\begin{equation}
[f,g,t] = F_{1}(x)
\end{equation}
where $\mathit{F_{1}}$ is the detector in the first stage structured with a
small CNN. The $\mathit{f}$ is face confidence score, $\mathit{t}$ is a vector
representing the prediction of bounding box regression, and
$\mathit{g}$ is orientation score.
The first objective, which is also the primary objective,
aims for distinguishing faces from non-faces with softmax
loss as follows:
\begin{equation}
L_{cls} = ylogf + (1-y)log(1-f)
\end{equation}
where $\mathit{y}$ equals 1 if $\mathit{x}$ is face, otherwise is 0.

\begin{table*}[!htbp]
\centering
\begin{tabular}{c|c|c|c}
\hline
\multirow{2}*{Method}&Recall rate at 100 FP on FDDB&Speed&\multirow{2}*{Model Size}\\
\cline{2-3}
&Up\qquad Down\qquad Left\qquad Right\qquad Ave&CPU\qquad GPU\\
\hline
Divide-and-Conquer &85.5\qquad 85.2\qquad 85.5\qquad 85.6\qquad 85.5&15FPS\qquad 20FPS&2.2M\\
Rotation Router &85.4\qquad 84.7\qquad 84.6\qquad 84.5\qquad 84.8 &12FPS\qquad 15FPS &2.5M\\
Cascade CNN &85.0\qquad 84.2\qquad 84.7\qquad 85.8\qquad 84.9 &31FPS\qquad 67FPS &4.2M\\
Cascade CNN &85.8\qquad 85.0\qquad 84.9\qquad 86.2\qquad 85.5 &16FPS\qquad 30FPS &4.7M\\
SSD500 &86.3\qquad 86.5\qquad 85.5\qquad 86.1\qquad 86.1 &1FPS\qquad 20FPS &95M\\
Faster R-CNN &84.2\qquad 82.5\qquad 81.9\qquad 82.1\qquad 82.7 &1FPS\qquad 20FPS &350M\\
Faster R-CNN &87.0\qquad 86.5\qquad 85.2\qquad 86.1\qquad 86.2 &0.5FPS\qquad 10FPS &547M\\
R-FCN &87.1\qquad 86.6\qquad 85.9\qquad 86.0\qquad 86.4 &0.8FPS\qquad 15FPS &123M\\
\hline
PCN (ours) &87.8\qquad 87.5\qquad 87.1\qquad 87.3\qquad 87.4\qquad &29FPS\qquad 63FPS &4.2M\\
\hline
\end{tabular}
\caption{Speed and accuracy comparison between different methods. The FDDB recall rate (\%) is at 100 false positives.}
\end{table*}

{\small
\bibliographystyle{ieee_fullname}
\bibliography{egbib}
}

\end{document}
