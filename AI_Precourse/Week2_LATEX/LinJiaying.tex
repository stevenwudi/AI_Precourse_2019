\documentclass[journal]{IEEEtran}


\ifCLASSINFOpdf

\else

\fi


\hyphenation{op-tical net-works semi-conduc-tor}


\begin{document}

\title{REAL TIME HUMAN FACE\\DETECTION AND TRACKING}


\author{Jatin Chatrath,
        Pankaj Gupta,
        Puneet Ahuja,
        Aryan Goel,
        Shaifali M.Arora% <-this % stops a space
\thanks{}% <-this % stops a space
\thanks{}% <-this % stops a space
\thanks{}}

% The paper headers


% make the title area
\maketitle

\begin{abstract}
This paper describes the technique for real time human face detection and tracking using a modified version of the algorithm suggested by Paul viola and Michael Jones. The paper starts with the introduction to human face detection and tracking, followed by apprehension of the Vila Jones algorithm and then discussing about the implementation in real video applications. Viola jones algorithm was based on object detection by extracting some specific features from the image. We used the same approach for real time human face detection and tracking. Simulation results of this developed algorithm shows the Real time human face detection and tracking supporting up to 50 human faces. This algorithm computes data and produce results in just a mere fraction of seconds.
\end{abstract}

% Note that keywords are not normally used for peerreview papers.
\begin{IEEEkeywords}
Human face detection, Integral Image, AdaBoost
\end{IEEEkeywords}

\IEEEpeerreviewmaketitle



\section{Introduction}

\IEEEPARstart{W}{ith} increasing terrorist activities and augmenting demand for video surveillance, it was the need of an hour to come up with an efficient and fast detection and tracking algorithm.

\subsection{RAPID HUMAN FACE DETECTION}
Algorithm used for face detection system consists of three major steps as discussed below.

\subsubsection{3.1. Integral Image Formation}
Rectangle features can be computed very rapidly using an intermediate representation for the image which we call the integral image.

\section{Conclusion}
This paper brings altogether a new algorithm, representations and insights which are generic and may well have broader application in computer region and image processing. Finally, this paper presents a set of detailed experiments on difficult face detection and tracking data set which has been widely studied. This data set includes faces under a wide range of conditions including: illumination, scale, pose and camera variation. Nevertheless the system which work under this algorithm are subjected to same set of conditions and but the algorithm is flexible enough to adjust according to the changing conditions.



The authors would like to thank...

\ifCLASSOPTIONcaptionsoff
  \newpage
\fi



\begin{IEEEbiography}{}

\end{IEEEbiography}

\end{document}

