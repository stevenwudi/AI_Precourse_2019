\documentclass{article}
%\usepackage[UTF-8]{ctex}中文
\usepackage{times}
\usepackage{epsfig}
\usepackage{graphicx}
\usepackage{amssymb}

\begin{document}
  
\title{\LaTeX\ Pose-Robust Face Recognition via Deep Residual Equivariant Mapping}
\author{ Cheng Li\\
\\
\\
{chengli@sensettime.com}
\and
Xiaoou Tang\\
\\
\\
{\tt\small xtang@ie.cuhk.edu.hk}
}
%\author{Kaidi Cao \and Yu Rong \and Cheng Li \and Xiaoou Tang \and Chen Change Loy

%}
\maketitle
\begin{abstract}
Face recognition achieves exceptional success thanks to the emergence of deep learning. However, many contemporary face recognition models still perform relatively poor in processing profile faces compared to frontal faces. A key reason is that the number of frontal and profile training faces are highly imbalanced - there are extensively more frontal training samples compared to profile ones. In addition, it is intrinsically hard to learn a deep representation that is geometrically invariant to large pose variations. In this study, we hypothesize that there is an inherent mapping between frontal and profile faces, and consequently, their discrepancy in the deep representation space can be bridged by an equivariant mapping. To exploit this mapping, we formulate a novel Deep Residual EquivAriant Mapping (DREAM) block, which is capable of adaptively adding residuals to the input deep representation to transform a profile face representation to a canonical pose that simplifies recognition. The DREAM block consistently enhances the performance of profile face recognition for many strong deep networks, including ResNet models, without deliberately augmenting training data of profile faces. The block is easy to use, light-weight, and can be implemented with a negligible computational overhead.
\end{abstract}


\begin{figure}
\begin{center}
%\fbox{\rule{0pt}{2in} \rule{0.9\linewidth}{0pt}}
   \includegraphics[width=0.8\linewidth]{fig.jpg}
\end{center}

\label{fig:long}
\label{fig:onecol}
\end{figure}

\begin{equation}
L_{cls} = ylogf + (1-y)log(1-f)
\end{equation}



\begin{table}
\begin{center}
\begin{tabular}{ccc}
\hline
Benchmark & Separate & Joint\\
\hline
AFW & 97.0\% & 98.7\% \\
\hline
FDDB & 89.7\% & 91.2\% \\
\hline
\end{tabular}
\end{center}
\end{table}

\end{document}
