\documentclass[10pt,twocolumn,letterpaper]{article}

\usepackage{cvpr}
\usepackage{times}
\usepackage{epsfig}
\usepackage{graphicx}
\usepackage{amsmath}
\usepackage{amssymb}

% Include other packages here, before hyperref.

% If you comment hyperref and then uncomment it, you should delete
% egpaper.aux before re-running latex.  (Or just hit 'q' on the first latex
% run, let it finish, and you should be clear).
\usepackage[pagebackref=true,breaklinks=true,letterpaper=true,colorlinks,bookmarks=false]{hyperref}

\cvprfinalcopy % *** Uncomment this line for the final submission

\def\cvprPaperID{****} % *** Enter the CVPR Paper ID here
\def\httilde{\mbox{\tt\raisebox{-.5ex}{\symbol{126}}}}

% Pages are numbered in submission mode, and unnumbered in camera-ready
\ifcvprfinal\pagestyle{empty}\fi
\begin{document}
%%%%%%%%% TITLE
\title{Learning Deep Models for Face Anti-Spoofing: Binary or Auxiliary Supervision }

\author{Yaojie Liu* \quad Amin Jourabloo* \quad Xiaoming Liu\\
Department of Computer Science and Engineering\\
Michigan State University,East Lansing MI 48824\\
{\tt\small \{liuyaoj1,jourablo,liuxm\}@msu.edu}
% For a paper whose authors are all at the same institution,
% omit the following lines up until the closing ``}''.
% Additional authors and addresses can be added with ``\and'',
% just like the second author.
% To save space, use either the email address or home page, not both
}



\maketitle
%\thispagestyle{empty}

%%%%%%%%% ABSTRACT
\begin{abstract}
   Face anti-spoofing is crucial to prevent face recognition
systems from a security breach. Previous deep learning ap-
proaches formulate face anti-spoofing as a binary classifi-
cation problem. Many of them struggle to grasp adequate
spoofing cues and generalize poorly. In this paper, we ar-
gue the importance of auxiliary supervision to guide the
learning toward discriminative and generalizable cues. A
CNN-RNN model is learned to estimate the face depth with
pixel-wise supervision, and to estimate rPPG signals with
sequence-wise supervision. The estimated depth and rPPG
are fused to distinguish live vs. spoof faces. Further, we
introduce a new face anti-spoofing database that covers a
large range of illumination, subject, and pose variations.
Experiments show that our model achieves the state-of-the-
art results on both intra- and cross-database testing.
\end{abstract}

%%%%%%%%% BODY TEXT
\section{Introduction}
With applications in phone unlock, access control, and
security, biometric systems are widely used in our daily
lives, and face is one of the most popular biometric modali-
ties. While face recognition systems gain popularity, attack-
ers present face spoofs (i.e., presentation attacks, PA) to the
system and attempt to be authenticated as the genuine user.
The face PA include printing a face on paper (print attack),
replaying a face video on a digital device (replay attack),
wearing a mask (mask attack), etc. To counteract PA, face
anti-spoofing techniques [~\cite{chetty2006multi},~\cite{frischholz2000biold},~\cite{frischholz2003avoiding},~\cite{li2004live}] 
are developed to
detect PA prior to a face image being recognized. There-
fore, face anti-spoofing is vital to ensure that face recogni-
tion systems are robust to PA and safe to use.\\
......
\section{One math formula,table and image}
\begin{equation}
\int \frac{d x}{\sqrt{x^2 \pm a^2}} =ln (x+ \sqrt{x^2 \pm a^2} )+C\\
\end{equation}



%-------------------------------------------------------------------------
\begin{table}
\begin{center}
\begin{tabular}{|l|c|c|c|c|}
\hline
Prot. & Method &APCER(\%)&BPCER(\%)&ACER(\%) \\
\hline\hline
&CPqD & 2.9&10.8&.9\\ \cline{2-5}
1&GRADIANT &1.3 &12.5&6.9 \\ \cline{2-5}
 &Proposed method&1.6&1.6&1.6\\
\hline
\end{tabular}
\end{center}
\caption{The intra-testing results on four protocols of Oulu.}
\end{table}


\begin{figure}[t]
\begin{center}
%\fbox{\rule{0pt}{2in} \rule{0.9\linewidth}{0pt}}
\includegraphics[width=0.8\linewidth]{figure.png}
\end{center}
   \caption{The statistics of the subjects in the SiW database. Left
		side: The histogram shows the distribution of the face sizes.}
\label{fig:long}
\label{fig:onecol}
\end{figure}



{\small
\bibliographystyle{ieee_fullname}
\bibliography{egbib}
}

\end{document}
